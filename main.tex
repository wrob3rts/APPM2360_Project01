\documentclass[12pt]{article}
\usepackage{amsmath,amssymb,amsthm}
\usepackage{graphicx}
\usepackage{geometry}
\geometry{top=1in, bottom=1.25in, left=1in, right=1in}
\usepackage{caption}
\usepackage{subcaption}
\usepackage{pdfpages}
\usepackage{hyperref}
\hypersetup{
    colorlinks=true, 
    linkcolor=black, 
    urlcolor=blue,
    pdftitle={Project 1: Heating and Cooling of Buildings},
    pdfauthor={William Roberts, Ryan Jenkins, Evan Miller}
}
\usepackage{booktabs}
\usepackage{array}
\usepackage{mathtools}
\usepackage{bm}
\usepackage{siunitx}
\usepackage{enumitem}
\setlist{nosep}

\title{Project 1: Heating and Cooling of Buildings}
\author{William Roberts \and Ryan Jenkins \and Evan Miller}
\date{\today}

\begin{document}
\maketitle

\tableofcontents
\clearpage

\section{Introduction}
Billy Bob works in a specialized laboratory dedicated to studying microbial cells, where maintaining precise temperature conditions is crucial. Each day, he monitors the building’s heating and cooling systems to ensure that the microbes are kept within safe temperature ranges. Even small fluctuations in temperature could compromise experiments or damage sensitive equipment. To better understand how the indoor environment responds to various influences, Billy Bob decides to investigate the factors that control the lab's temperature. He considers the impact of the outside weather, the heat generated by internal sources, and the role of the thermostat in maintaining a stable environment for the laboratory. By combining analytical modeling with numerical simulations, he aims to predict temperature changes, identify potential risks, and develop strategies to keep the lab conditions safe and consistent.

\section{Background}
For the purposes of modeling the temperature of the building, $T$, at time $t$. The three elements Billy Bob is looking at are: the ambient outside temperature, $A(t)$, the heat produced by machinery, lights, and people, $H(t)$, and artificial heating and cooling, $Q(t)$, yielding the differential equation $ \frac{dT}{dt} = A(t) + H(t) + Q(t)$.

\section{Analytical investigation (Task Set A)}

\section{Numerical methods and verification (Task Set B)}

\section{Refined models and simulations}
\subsection{Varying outside temperature (Task Set C)}
\subsection{Internal heat sources (Task Set D)}
\subsection{Thermostat control (Task Set E)}

\section{Combined scenarios and safety analysis (Task Set F)}

\clearpage
\section{Results}

\section{Conclusion}

\clearpage
\appendix
\section{Appendix A: Code}
%\includepdf[pages=-]{code.pdf}

\section{Appendix B: Lengthy Calculations}


\end{document}




